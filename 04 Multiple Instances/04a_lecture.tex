\documentclass[ignorenonframetext,,t]{beamer}

\usepackage{listings}

\setbeamertemplate{caption}[numbered]
\setbeamertemplate{caption label separator}{: }
\setbeamercolor{caption name}{fg=normal text.fg}
\setbeamerfont{caption}{size=\tiny, shape=\itshape}
\beamertemplatenavigationsymbolsempty


  \usepackage{lmodern}


\usepackage{amssymb,amsmath}
\usepackage{ifxetex,ifluatex}
\usepackage{fixltx2e} % provides \textsubscript
\ifnum 0\ifxetex 1\fi\ifluatex 1\fi=0 % if pdftex

\usepackage[T1]{fontenc}
\usepackage[utf8]{inputenc}
\usepackage{eurosym}

\else % if luatex or xelatex
  \ifxetex
    \usepackage{mathspec}
  \else
    \usepackage{fontspec}
\fi

\defaultfontfeatures{Ligatures=TeX,Scale=MatchLowercase}






\fi


%%%%%%%%%%%%%%%%%%%%%%%%%%%%%%%%%%%%%%%%%%%%%%%%%%%%%%%%%%%%%%%%%%%%%%%%%%%%%%%%
%%%%%%%%%%%%%%%%%%%%%%%%%%%%%%%%%%%%%%%%%%%%%%%%%%%%%%%%%%%%%%%%%%%%%%%%%%%%%%%%
% Font sizes
%\usefonttheme[stillsansseriflarge]{serif}
\setbeamerfont{frametitle}{series=\bfseries}
\setbeamerfont{title}{series=\bfseries,size=\Large}
\setbeamerfont{subtitle}{size=\small}


% Color
\usepackage{color}
\definecolor{spamwell}{RGB}{0,126,63}
\setbeamercolor{frametitle}{fg=spamwell,bg=spamwell!20}
\setbeamercolor{title}{fg=spamwell,bg=spamwell!20}
\setbeamercolor{structure}{fg=spamwell}
\setbeamercolor{alerted text}{fg=spamwell}
%\setbeamercolor{bibliography entry author}{fg=red!20}

\hypersetup{
  colorlinks,
  allcolors=spamwell!80}

% Title graphic
\titlegraphic{\includegraphics[width=2cm]{~/ownCloud/img/logo-uni}\hspace{1cm}\includegraphics[width=2cm]{~/ownCloud/img/logo-abteilung}}

% Itemize list
\setbeamertemplate{itemize items}[circle]

\let\oldtextbf\textbf
\renewcommand{\textbf}[1]{\textcolor{spamwell}{\oldtextbf{#1}}}


\usepackage{pgf}

% Booktabs
\usepackage{booktabs}
\usepackage{longtable}
\usepackage{array}
\usepackage{multirow}
\usepackage{wrapfig}
\usepackage{float}
\usepackage{colortbl}
\usepackage{pdflscape}
\usepackage{tabu}
\usepackage{threeparttable}

% Section


% Footline
\setbeamercolor{footlinecolor}{fg=spamwell, bg=spamwell!20}

\setbeamertemplate{footline}{%
  \begin{beamercolorbox}[sep=1em,wd=\paperwidth,leftskip=0.2cm,rightskip=0.2cm]{footlinecolor}
    \hspace{0.01cm}%
    %Johannes Signer (jsigner@gwdg.de)\hfill\insertframenumber/\inserttotalframenumber
    Johannes Signer (jsigner@uni-goettingen.de)\hfill\insertframenumber
  \end{beamercolorbox}%
}

\setbeamertemplate{navigation symbols}{}

% Verbatim
\usepackage{xcolor, url}
\definecolor{dark-gray}{gray}{0.30}
\makeatletter
\renewcommand\verbatim@font{\footnotesize\color{dark-gray}\normalfont\ttfamily}
\makeatletter

% tight list

\providecommand{\tightlist}{%
\setlength{\itemsep}{0pt}\setlength{\parskip}{0pt}}

% use upquote if available, for straight quotes in verbatim environments
\IfFileExists{upquote.sty}{\usepackage{upquote}}{}
% use microtype if available
\IfFileExists{microtype.sty}{%
\usepackage{microtype}
\UseMicrotypeSet[protrusion]{basicmath} % disable protrusion for tt fonts
}{}


\usepackage{copyrightbox}

\newif\ifbibliography



\usepackage{color}
\usepackage{fancyvrb}
\newcommand{\VerbBar}{|}
\newcommand{\VERB}{\Verb[commandchars=\\\{\}]}
\DefineVerbatimEnvironment{Highlighting}{Verbatim}{commandchars=\\\{\}}
% Add ',fontsize=\small' for more characters per line
\usepackage{framed}
\definecolor{shadecolor}{RGB}{248,248,248}
\newenvironment{Shaded}{\begin{snugshade}}{\end{snugshade}}
\newcommand{\AlertTok}[1]{\textcolor[rgb]{0.94,0.16,0.16}{#1}}
\newcommand{\AnnotationTok}[1]{\textcolor[rgb]{0.56,0.35,0.01}{\textbf{\textit{#1}}}}
\newcommand{\AttributeTok}[1]{\textcolor[rgb]{0.13,0.29,0.53}{#1}}
\newcommand{\BaseNTok}[1]{\textcolor[rgb]{0.00,0.00,0.81}{#1}}
\newcommand{\BuiltInTok}[1]{#1}
\newcommand{\CharTok}[1]{\textcolor[rgb]{0.31,0.60,0.02}{#1}}
\newcommand{\CommentTok}[1]{\textcolor[rgb]{0.56,0.35,0.01}{\textit{#1}}}
\newcommand{\CommentVarTok}[1]{\textcolor[rgb]{0.56,0.35,0.01}{\textbf{\textit{#1}}}}
\newcommand{\ConstantTok}[1]{\textcolor[rgb]{0.56,0.35,0.01}{#1}}
\newcommand{\ControlFlowTok}[1]{\textcolor[rgb]{0.13,0.29,0.53}{\textbf{#1}}}
\newcommand{\DataTypeTok}[1]{\textcolor[rgb]{0.13,0.29,0.53}{#1}}
\newcommand{\DecValTok}[1]{\textcolor[rgb]{0.00,0.00,0.81}{#1}}
\newcommand{\DocumentationTok}[1]{\textcolor[rgb]{0.56,0.35,0.01}{\textbf{\textit{#1}}}}
\newcommand{\ErrorTok}[1]{\textcolor[rgb]{0.64,0.00,0.00}{\textbf{#1}}}
\newcommand{\ExtensionTok}[1]{#1}
\newcommand{\FloatTok}[1]{\textcolor[rgb]{0.00,0.00,0.81}{#1}}
\newcommand{\FunctionTok}[1]{\textcolor[rgb]{0.13,0.29,0.53}{\textbf{#1}}}
\newcommand{\ImportTok}[1]{#1}
\newcommand{\InformationTok}[1]{\textcolor[rgb]{0.56,0.35,0.01}{\textbf{\textit{#1}}}}
\newcommand{\KeywordTok}[1]{\textcolor[rgb]{0.13,0.29,0.53}{\textbf{#1}}}
\newcommand{\NormalTok}[1]{#1}
\newcommand{\OperatorTok}[1]{\textcolor[rgb]{0.81,0.36,0.00}{\textbf{#1}}}
\newcommand{\OtherTok}[1]{\textcolor[rgb]{0.56,0.35,0.01}{#1}}
\newcommand{\PreprocessorTok}[1]{\textcolor[rgb]{0.56,0.35,0.01}{\textit{#1}}}
\newcommand{\RegionMarkerTok}[1]{#1}
\newcommand{\SpecialCharTok}[1]{\textcolor[rgb]{0.81,0.36,0.00}{\textbf{#1}}}
\newcommand{\SpecialStringTok}[1]{\textcolor[rgb]{0.31,0.60,0.02}{#1}}
\newcommand{\StringTok}[1]{\textcolor[rgb]{0.31,0.60,0.02}{#1}}
\newcommand{\VariableTok}[1]{\textcolor[rgb]{0.00,0.00,0.00}{#1}}
\newcommand{\VerbatimStringTok}[1]{\textcolor[rgb]{0.31,0.60,0.02}{#1}}
\newcommand{\WarningTok}[1]{\textcolor[rgb]{0.56,0.35,0.01}{\textbf{\textit{#1}}}}




% Prevent slide breaks in the middle of a paragraph:
\widowpenalties 1 10000
\raggedbottom

\AtBeginPart{
\let\insertpartnumber\relax
\let\partname\relax
\frame{\partpage}
}

% Thanks to http://tex.stackexchange.com/questions/178800/creating-sections-each-with-title-pages-in-beamers-slides
\AtBeginSection[]{
  \begin{frame}
  \vfill
  \centering
  \begin{beamercolorbox}[sep=8pt,center,shadow=true,rounded=false]{title}
    \usebeamerfont{title}\insertsectionhead\par%
  \end{beamercolorbox}
  \vfill
%  \tableofcontents[currentsection]
  \end{frame}
}



\AtBeginSubsection{
\let\insertsubsectionnumber\relax
\let\subsectionname\relax
\frame{\subsectionpage}
}


% Block titles

%\setbeamertemplate{blocktitle}{%
%    \usebeamerfont{blocktitle}\insertblocktitle%
%    \vphantom{g}% To avoid fluctuations per frame
%    %\hrule% Uncomment to see desired effect, without a full-width hrule
%    \makebox[\linewidth][l]{\rule{\paperwidth}{0.4pt}}%
%}
% https://tex.stackexchange.com/questions/524621/customize-beamer-block-environment-underline-block-title
\addtobeamertemplate{block begin}{%
    \let\oldinsertblocktitle\insertblocktitle%
    \def\insertblocktitle{\underline{\oldinsertblocktitle}}%
}{}





\setlength{\emergencystretch}{3em}  % prevent overfull lines
\providecommand{\tightlist}{%
\setlength{\itemsep}{0pt}\setlength{\parskip}{0pt}}

\setcounter{secnumdepth}{0}


%  itemsep for lists: https://stackoverflow.com/questions/58318568/how-to-increase-distance-between-bullet-points-in-r-markdown
\renewcommand{\tightlist}{\setlength{\itemsep}{1.4ex}\setlength{\parskip}{0pt}}


% footnotesize
\setbeamerfont{footnote}{size=\tiny}

% title
\title{Working with multiple instances}


\author{Johannes Signer}


\date{January 2024}

\begin{document}
\frame{\titlepage}



\begin{frame}[fragile]{Motivation}
\protect\hypertarget{motivation}{}
\begin{itemize}
\tightlist
\item
  Most telemetry studies deal with more than one animal.
\item
  Dealing with multiple animals can complicate analyses significantly.
\item
  Some packages provide an infrastructure to deal with multiple animals
  (e.g.~\texttt{adehabitat}, \texttt{move}).
\item
  \texttt{amt} has very limited support for dealing with multiple
  animals, but relies on the infrastructure provided by the
  \texttt{purrr} package. The hope is, that this adds additional
  flexibility.
\end{itemize}
\end{frame}

\begin{frame}{Examples}
\protect\hypertarget{examples}{}
\begin{itemize}
\tightlist
\item
  Weekly home-range size of bears.
\item
  Does home-range size differ between different sexes, treatments?
\item
  Individual-level habitat selection?
\item
  What is the mean step length for different animals during day and
  night?
\end{itemize}
\end{frame}

\begin{frame}[fragile]{Repeating tasks in R}
\protect\hypertarget{repeating-tasks-in-r}{}
Three possible ways to deal with multiple instances in R:

\begin{enumerate}
\tightlist
\item
  Copy and paste your code and make slight adaptions (\textbf{don't do
  this}).
\item
  Use a \texttt{for}-loop
\item
  Use functional programming
\end{enumerate}
\end{frame}

\begin{frame}[fragile]
Let's consider the following problem: We want to calculate the mean step
length of each fisher from the \texttt{amt\_fisher} data set:

\begin{Shaded}
\begin{Highlighting}[]
\FunctionTok{library}\NormalTok{(amt)}
\FunctionTok{data}\NormalTok{(amt\_fisher)}
\FunctionTok{unique}\NormalTok{(amt\_fisher}\SpecialCharTok{$}\NormalTok{name)}
\end{Highlighting}
\end{Shaded}

\begin{verbatim}
[1] "Leroy"   "Ricky T" "Lupe"    "Lucile" 
\end{verbatim}
\end{frame}

\begin{frame}[fragile]
For the first animal \texttt{Leroy}\footnote<.->{Note, I am ignoring
  sampling rates here.}

\begin{Shaded}
\begin{Highlighting}[]
\NormalTok{amt\_fisher }\SpecialCharTok{|\textgreater{}} \FunctionTok{filter}\NormalTok{(name }\SpecialCharTok{==} \StringTok{"Leroy"}\NormalTok{) }\SpecialCharTok{|\textgreater{}} 
  \FunctionTok{step\_lengths}\NormalTok{() }\SpecialCharTok{|\textgreater{}} \FunctionTok{mean}\NormalTok{(}\AttributeTok{na.rm =} \ConstantTok{TRUE}\NormalTok{)}
\end{Highlighting}
\end{Shaded}

\begin{verbatim}
[1] 186.9954
\end{verbatim}
\end{frame}

\begin{frame}[fragile]
Now we could do the same for the other three individuals:

\begin{Shaded}
\begin{Highlighting}[]
\NormalTok{amt\_fisher }\SpecialCharTok{|\textgreater{}} \FunctionTok{filter}\NormalTok{(name }\SpecialCharTok{==} \StringTok{"Ricky T"}\NormalTok{) }\SpecialCharTok{|\textgreater{}} 
  \FunctionTok{step\_lengths}\NormalTok{() }\SpecialCharTok{|\textgreater{}} \FunctionTok{mean}\NormalTok{(}\AttributeTok{na.rm =} \ConstantTok{TRUE}\NormalTok{)}
\end{Highlighting}
\end{Shaded}

\begin{verbatim}
[1] 41.9739
\end{verbatim}

\begin{Shaded}
\begin{Highlighting}[]
\NormalTok{amt\_fisher }\SpecialCharTok{|\textgreater{}} \FunctionTok{filter}\NormalTok{(name }\SpecialCharTok{==} \StringTok{"Lupe"}\NormalTok{) }\SpecialCharTok{|\textgreater{}} 
  \FunctionTok{step\_lengths}\NormalTok{() }\SpecialCharTok{|\textgreater{}} \FunctionTok{mean}\NormalTok{(}\AttributeTok{na.rm =} \ConstantTok{TRUE}\NormalTok{)}
\end{Highlighting}
\end{Shaded}

\begin{verbatim}
[1] 61.04558
\end{verbatim}

\begin{Shaded}
\begin{Highlighting}[]
\NormalTok{amt\_fisher }\SpecialCharTok{|\textgreater{}} \FunctionTok{filter}\NormalTok{(name }\SpecialCharTok{==} \StringTok{"Lucile"}\NormalTok{) }\SpecialCharTok{|\textgreater{}} 
  \FunctionTok{step\_lengths}\NormalTok{() }\SpecialCharTok{|\textgreater{}} \FunctionTok{mean}\NormalTok{(}\AttributeTok{na.rm =} \ConstantTok{TRUE}\NormalTok{)}
\end{Highlighting}
\end{Shaded}

\begin{verbatim}
[1] 85.95821
\end{verbatim}
\end{frame}

\begin{frame}[fragile]
\begin{itemize}
\item
  We would also need an other vector to save the results.
\item
  While this approach might be OK for only a few individuals, it becomes
  very tedious for many animals or if you would like to add an
  additional grouping factor (say, we would also want to calculate the
  mean step for each day).
\item
  \texttt{for}-loops can be useful here.
\end{itemize}
\end{frame}

\begin{frame}[fragile]
Note, only the name of the animal changes

\begin{Shaded}
\begin{Highlighting}[]
\NormalTok{amt\_fisher }\SpecialCharTok{|\textgreater{}} \FunctionTok{filter}\NormalTok{(name }\SpecialCharTok{==} \StringTok{"Ricky T"}\NormalTok{) }\SpecialCharTok{|\textgreater{}} 
  \FunctionTok{step\_lengths}\NormalTok{() }\SpecialCharTok{|\textgreater{}} \FunctionTok{mean}\NormalTok{(}\AttributeTok{na.rm =} \ConstantTok{TRUE}\NormalTok{)}
\end{Highlighting}
\end{Shaded}

\begin{verbatim}
[1] 41.9739
\end{verbatim}

\begin{Shaded}
\begin{Highlighting}[]
\NormalTok{amt\_fisher }\SpecialCharTok{|\textgreater{}} \FunctionTok{filter}\NormalTok{(name }\SpecialCharTok{==} \StringTok{"Lupe"}\NormalTok{) }\SpecialCharTok{|\textgreater{}} 
  \FunctionTok{step\_lengths}\NormalTok{() }\SpecialCharTok{|\textgreater{}} \FunctionTok{mean}\NormalTok{(}\AttributeTok{na.rm =} \ConstantTok{TRUE}\NormalTok{)}
\end{Highlighting}
\end{Shaded}

\begin{verbatim}
[1] 61.04558
\end{verbatim}

\begin{Shaded}
\begin{Highlighting}[]
\NormalTok{amt\_fisher }\SpecialCharTok{|\textgreater{}} \FunctionTok{filter}\NormalTok{(name }\SpecialCharTok{==} \StringTok{"Lucile"}\NormalTok{) }\SpecialCharTok{|\textgreater{}} 
  \FunctionTok{step\_lengths}\NormalTok{() }\SpecialCharTok{|\textgreater{}} \FunctionTok{mean}\NormalTok{(}\AttributeTok{na.rm =} \ConstantTok{TRUE}\NormalTok{)}
\end{Highlighting}
\end{Shaded}

\begin{verbatim}
[1] 85.95821
\end{verbatim}
\end{frame}

\begin{frame}[fragile]
We could use a variable to store the name of animal currently under
evaluation, for example \texttt{i}:

\begin{Shaded}
\begin{Highlighting}[]
\NormalTok{i }\OtherTok{\textless{}{-}} \StringTok{"Ricky T"}
\NormalTok{amt\_fisher }\SpecialCharTok{|\textgreater{}} \FunctionTok{filter}\NormalTok{(name }\SpecialCharTok{==}\NormalTok{ i) }\SpecialCharTok{|\textgreater{}} 
  \FunctionTok{step\_lengths}\NormalTok{() }\SpecialCharTok{|\textgreater{}} \FunctionTok{mean}\NormalTok{(}\AttributeTok{na.rm =} \ConstantTok{TRUE}\NormalTok{)}
\end{Highlighting}
\end{Shaded}

\begin{verbatim}
[1] 41.9739
\end{verbatim}

\begin{Shaded}
\begin{Highlighting}[]
\NormalTok{i }\OtherTok{\textless{}{-}} \StringTok{"Lupe"}
\NormalTok{amt\_fisher }\SpecialCharTok{|\textgreater{}} \FunctionTok{filter}\NormalTok{(name }\SpecialCharTok{==}\NormalTok{ i) }\SpecialCharTok{|\textgreater{}} 
  \FunctionTok{step\_lengths}\NormalTok{() }\SpecialCharTok{|\textgreater{}} \FunctionTok{mean}\NormalTok{(}\AttributeTok{na.rm =} \ConstantTok{TRUE}\NormalTok{)}
\end{Highlighting}
\end{Shaded}

\begin{verbatim}
[1] 61.04558
\end{verbatim}

\begin{Shaded}
\begin{Highlighting}[]
\NormalTok{i }\OtherTok{\textless{}{-}} \StringTok{"Lucile"}
\NormalTok{amt\_fisher }\SpecialCharTok{|\textgreater{}} \FunctionTok{filter}\NormalTok{(name }\SpecialCharTok{==}\NormalTok{ i) }\SpecialCharTok{|\textgreater{}} 
  \FunctionTok{step\_lengths}\NormalTok{() }\SpecialCharTok{|\textgreater{}} \FunctionTok{mean}\NormalTok{(}\AttributeTok{na.rm =} \ConstantTok{TRUE}\NormalTok{)}
\end{Highlighting}
\end{Shaded}

\begin{verbatim}
[1] 85.95821
\end{verbatim}
\end{frame}

\begin{frame}[fragile]
\begin{itemize}
\tightlist
\item
  \texttt{i} takes the name of the animal that is currently under
  evaluation. Once the code block is executed, \texttt{i} is updated to
  the next name.
\item
  Everything else is the \textbf{same}.
\end{itemize}

A \texttt{for}-loop has one variable, that changes for each go.

\begin{Shaded}
\begin{Highlighting}[]
\ControlFlowTok{for}\NormalTok{ (i }\ControlFlowTok{in} \FunctionTok{c}\NormalTok{(}\StringTok{"Leroy"}\NormalTok{, }\StringTok{"Ricky T"}\NormalTok{, }\StringTok{"Lupe"}\NormalTok{, }\StringTok{"Lucile"}\NormalTok{)) \{}
  \CommentTok{\# Do something}
\NormalTok{\}}
\end{Highlighting}
\end{Shaded}

The body of the looped (everything between \texttt{\{} and \texttt{\}})
is executed \textbf{four} times. Each time the value of \texttt{i}
changes.
\end{frame}

\begin{frame}[fragile]
We can use such a loop to calculate the mean step length for each
animal:

\begin{Shaded}
\begin{Highlighting}[]
\ControlFlowTok{for}\NormalTok{ (i }\ControlFlowTok{in} \FunctionTok{c}\NormalTok{(}\StringTok{"Leroy"}\NormalTok{, }\StringTok{"Ricky T"}\NormalTok{, }\StringTok{"Lupe"}\NormalTok{, }\StringTok{"Lucile"}\NormalTok{)) \{}
\NormalTok{  amt\_fisher }\SpecialCharTok{|\textgreater{}} \FunctionTok{filter}\NormalTok{(name }\SpecialCharTok{==}\NormalTok{ i) }\SpecialCharTok{|\textgreater{}} 
    \FunctionTok{step\_lengths}\NormalTok{() }\SpecialCharTok{|\textgreater{}} \FunctionTok{mean}\NormalTok{(}\AttributeTok{na.rm =} \ConstantTok{TRUE}\NormalTok{)}
\NormalTok{\}}
\end{Highlighting}
\end{Shaded}
\end{frame}

\begin{frame}[fragile]
Finally, we have to take care of the results and save them in a new
variable, which I called \texttt{res} here.

\begin{Shaded}
\begin{Highlighting}[]
\NormalTok{res }\OtherTok{\textless{}{-}} \FunctionTok{c}\NormalTok{()}
\NormalTok{j }\OtherTok{\textless{}{-}} \DecValTok{1}
\ControlFlowTok{for}\NormalTok{ (i }\ControlFlowTok{in} \FunctionTok{c}\NormalTok{(}\StringTok{"Leroy"}\NormalTok{, }\StringTok{"Ricky T"}\NormalTok{, }\StringTok{"Lupe"}\NormalTok{, }\StringTok{"Lucile"}\NormalTok{)) \{}
\NormalTok{  res[j] }\OtherTok{\textless{}{-}}\NormalTok{ amt\_fisher }\SpecialCharTok{|\textgreater{}} \FunctionTok{filter}\NormalTok{(name }\SpecialCharTok{==}\NormalTok{ i) }\SpecialCharTok{|\textgreater{}} 
    \FunctionTok{step\_lengths}\NormalTok{() }\SpecialCharTok{|\textgreater{}} \FunctionTok{mean}\NormalTok{(}\AttributeTok{na.rm =} \ConstantTok{TRUE}\NormalTok{)}
\NormalTok{  j }\OtherTok{\textless{}{-}}\NormalTok{ j }\SpecialCharTok{+} \DecValTok{1}
\NormalTok{\}}
\end{Highlighting}
\end{Shaded}
\end{frame}

\begin{frame}[fragile]
\begin{itemize}
\tightlist
\item
  \texttt{for}-loops are a significant improvement compared to approach
  \#1.
\item
  However, \texttt{for}-loops \emph{can} become a bit tedious if we have
  multiple grouping instances (e.g., multiple animals for multiple
  years). We potentially need nested \texttt{for}-loops.
\end{itemize}
\end{frame}

\begin{frame}[fragile]{A slightly deeper look at R}
\protect\hypertarget{a-slightly-deeper-look-at-r}{}
\begin{block}{What is a \texttt{tibbles}?}
\protect\hypertarget{what-is-a-tibbles}{}
\begin{itemize}
\tightlist
\item
  \texttt{tibbles} are \emph{modern} \texttt{data.frame}s.
\item
  A tibble can have list columns.
\item
  A list is an other data structure in R, that can hold any other data
  structure.
\end{itemize}

With list columns it is easy to have more complex splits of your data
(e.g., animals/seasons/weeks).
\end{block}
\end{frame}

\begin{frame}[fragile]
\begin{block}{What is a list?}
\protect\hypertarget{what-is-a-list}{}
Lists are yet an other data structures for R. Lists are can contain any
object (even other lists).

\begin{Shaded}
\begin{Highlighting}[]
\NormalTok{l }\OtherTok{\textless{}{-}} \FunctionTok{list}\NormalTok{(}\StringTok{"a"}\NormalTok{, }\AttributeTok{b =} \DecValTok{1}\SpecialCharTok{:}\DecValTok{10}\NormalTok{, }\AttributeTok{x =} \FunctionTok{list}\NormalTok{(}\FunctionTok{list}\NormalTok{(}\DecValTok{1}\SpecialCharTok{:}\DecValTok{10}\NormalTok{)))}
\FunctionTok{str}\NormalTok{(l)}
\end{Highlighting}
\end{Shaded}

\begin{verbatim}
List of 3
$ : chr "a"
$ b: int [1:10] 1 2 3 4 5 6 7 8 9 10
$ x:List of 1
..$ :List of 1
.. ..$ : int [1:10] 1 2 3 4 5 6 7 8 9 10
\end{verbatim}
\end{block}
\end{frame}

\begin{frame}[fragile]
\begin{block}{Examples for lists}
\protect\hypertarget{examples-for-lists}{}
\begin{Shaded}
\begin{Highlighting}[]
\NormalTok{x }\OtherTok{\textless{}{-}} \FunctionTok{list}\NormalTok{(}\AttributeTok{a =} \DecValTok{1}\SpecialCharTok{:}\DecValTok{3}\NormalTok{, }\AttributeTok{b =} \FunctionTok{list}\NormalTok{(}\DecValTok{1}\SpecialCharTok{:}\DecValTok{3}\NormalTok{))}
\NormalTok{x[[}\DecValTok{1}\NormalTok{]]}
\end{Highlighting}
\end{Shaded}

\begin{verbatim}
[1] 1 2 3
\end{verbatim}

\begin{Shaded}
\begin{Highlighting}[]
\NormalTok{x}\SpecialCharTok{$}\NormalTok{a}
\end{Highlighting}
\end{Shaded}

\begin{verbatim}
[1] 1 2 3
\end{verbatim}

\begin{Shaded}
\begin{Highlighting}[]
\NormalTok{x[[}\StringTok{"b"}\NormalTok{]]}
\end{Highlighting}
\end{Shaded}

\begin{verbatim}
[[1]]
[1] 1 2 3
\end{verbatim}
\end{block}
\end{frame}

\begin{frame}{Functional programming in R}
\protect\hypertarget{functional-programming-in-r}{}
\begin{block}{What is functional programming?}
\protect\hypertarget{what-is-functional-programming}{}
\begin{quote}
Simply put, FP is exactly what it sounds like. If you are doing
something more than once, it belongs in a function. In FP, functions are
the primary method with which you should carry out tasks. All actions
are just (often creative) implementations of functions you've written.
\href{https://towardsdatascience.com/cleaner-r-code-with-functional-programming-adc37931ef7a}{towardsdatascience.com}
\end{quote}
\end{block}
\end{frame}

\begin{frame}[fragile]
\begin{block}{The \texttt{apply}-family}
\protect\hypertarget{the-apply-family}{}
These functions apply a function on a \texttt{matrix}, \texttt{vector}
or \texttt{list}.

For matrices:

\begin{itemize}
\tightlist
\item
  \texttt{apply()} (we won't cover this here in more detail)
\end{itemize}

For vectors and lists

\begin{itemize}
\tightlist
\item
  \texttt{lapply()}
\item
  \texttt{sapply()}
\end{itemize}
\end{block}
\end{frame}

\begin{frame}[fragile]
\begin{block}{\texttt{lapply()}}
\protect\hypertarget{lapply}{}
\texttt{lapply()} applies a function to each element of a list (or a
vector) and returns list, then \texttt{lapply} can be used.

\begin{Shaded}
\begin{Highlighting}[]
\NormalTok{l }\OtherTok{\textless{}{-}} \FunctionTok{list}\NormalTok{(}\DecValTok{1}\SpecialCharTok{:}\DecValTok{3}\NormalTok{, }\DecValTok{2}\NormalTok{)}
\FunctionTok{lapply}\NormalTok{(l, length)}
\end{Highlighting}
\end{Shaded}

\begin{verbatim}
[[1]]
[1] 3

[[2]]
[1] 1
\end{verbatim}
\end{block}
\end{frame}

\begin{frame}[fragile]
We could achieve the same with a \texttt{for}-loop:

\begin{Shaded}
\begin{Highlighting}[]
\NormalTok{res }\OtherTok{\textless{}{-}} \FunctionTok{list}\NormalTok{()}
\ControlFlowTok{for}\NormalTok{ (i }\ControlFlowTok{in} \DecValTok{1}\SpecialCharTok{:}\DecValTok{2}\NormalTok{) \{}
\NormalTok{  res[[i]] }\OtherTok{\textless{}{-}} \FunctionTok{length}\NormalTok{(l[[i]])}
\NormalTok{\}}
\end{Highlighting}
\end{Shaded}
\end{frame}

\begin{frame}[fragile]
Here it would make more sense to use \texttt{sapply} (R will try to
simplify the data structure of the result).

\begin{Shaded}
\begin{Highlighting}[]
\FunctionTok{sapply}\NormalTok{(l, length)}
\end{Highlighting}
\end{Shaded}

\begin{verbatim}
[1] 3 1
\end{verbatim}
\end{frame}

\begin{frame}[fragile]
Note, we used the shortest possible way, it is also possible to
explicitly work with the object the function is applied to.

\begin{Shaded}
\begin{Highlighting}[]
\FunctionTok{sapply}\NormalTok{(l, }\ControlFlowTok{function}\NormalTok{(x) }\FunctionTok{length}\NormalTok{(x))}
\end{Highlighting}
\end{Shaded}

\begin{verbatim}
[1] 3 1
\end{verbatim}

Since \texttt{length} only uses one argument (here \texttt{x}), we do
not have to explicitly call the function.

This can be shorted (since R 4.1) to

\begin{Shaded}
\begin{Highlighting}[]
\FunctionTok{sapply}\NormalTok{(l, \textbackslash{}(x) }\FunctionTok{length}\NormalTok{(x))}
\end{Highlighting}
\end{Shaded}

\begin{verbatim}
[1] 3 1
\end{verbatim}
\end{frame}

\begin{frame}[fragile]
The \texttt{purrr} package provides a more type-stable way to
\texttt{*apply()} functions. These are called \texttt{map\_*()}.

\begin{itemize}
\tightlist
\item
  \texttt{lapply()} -\textgreater{} \texttt{map()}
\item
  \texttt{sapply()} -\textgreater{} \texttt{map\_*()}

  \begin{itemize}
  \tightlist
  \item
    \texttt{map\_lgl()} for logical values
  \item
    \texttt{map\_dbl()} for doubles
  \item
    \texttt{map\_int()} for integers
  \item
    \texttt{map\_chr()} for text
  \end{itemize}
\end{itemize}

In addition there variants of all \texttt{map*()} functions that take
two inputs (\texttt{map2\_*()}) and many inputs (\texttt{pmap\_*()}).
\end{frame}

\begin{frame}[fragile]
\begin{Shaded}
\begin{Highlighting}[]
\FunctionTok{library}\NormalTok{(purrr)}
\FunctionTok{map}\NormalTok{(l, length)}
\end{Highlighting}
\end{Shaded}

\begin{verbatim}
[[1]]
[1] 3

[[2]]
[1] 1
\end{verbatim}

Better

\begin{Shaded}
\begin{Highlighting}[]
\FunctionTok{map\_int}\NormalTok{(l, length)}
\end{Highlighting}
\end{Shaded}

\begin{verbatim}
[1] 3 1
\end{verbatim}
\end{frame}

\begin{frame}[fragile]
Again, it is possible to access the object directly. This can be done as
before with \texttt{function(\textless{}var\textgreater{})},
\texttt{\textbackslash{}(x)} or \texttt{\textasciitilde{}}. When using
\texttt{\textasciitilde{}} the object currently under evaluation can be
accessed with \texttt{.} or \texttt{.x}.

\begin{Shaded}
\begin{Highlighting}[]
\FunctionTok{map\_int}\NormalTok{(l, }\ControlFlowTok{function}\NormalTok{(x) }\FunctionTok{length}\NormalTok{(x))}
\end{Highlighting}
\end{Shaded}

\begin{verbatim}
[1] 3 1
\end{verbatim}

\begin{Shaded}
\begin{Highlighting}[]
\FunctionTok{map\_int}\NormalTok{(l, }\SpecialCharTok{\textasciitilde{}} \FunctionTok{length}\NormalTok{(.))}
\end{Highlighting}
\end{Shaded}

\begin{verbatim}
[1] 3 1
\end{verbatim}

\begin{Shaded}
\begin{Highlighting}[]
\FunctionTok{map\_int}\NormalTok{(l, }\SpecialCharTok{\textasciitilde{}} \FunctionTok{length}\NormalTok{(.x))}
\end{Highlighting}
\end{Shaded}

\begin{verbatim}
[1] 3 1
\end{verbatim}
\end{frame}

\begin{frame}[fragile]
An example for \texttt{map2\_*}:

\begin{Shaded}
\begin{Highlighting}[]
\NormalTok{a }\OtherTok{\textless{}{-}} \DecValTok{1}\SpecialCharTok{:}\DecValTok{4}
\NormalTok{b }\OtherTok{\textless{}{-}} \DecValTok{4}\SpecialCharTok{:}\DecValTok{1}

\FunctionTok{map2\_dbl}\NormalTok{(a, b, }\SpecialCharTok{\textasciitilde{}}\NormalTok{ .x }\SpecialCharTok{+}\NormalTok{ .y)}
\end{Highlighting}
\end{Shaded}

\begin{verbatim}
[1] 5 5 5 5
\end{verbatim}
\end{frame}

\begin{frame}[fragile]{Nest and unnest}
\protect\hypertarget{nest-and-unnest}{}
An example data set

\begin{Shaded}
\begin{Highlighting}[]
\FunctionTok{set.seed}\NormalTok{(}\DecValTok{12}\NormalTok{)}
\NormalTok{dat }\OtherTok{\textless{}{-}} \FunctionTok{data.frame}\NormalTok{(}
  \AttributeTok{id =} \FunctionTok{rep}\NormalTok{(}\DecValTok{1}\SpecialCharTok{:}\DecValTok{10}\NormalTok{, }\AttributeTok{each =} \DecValTok{10}\NormalTok{), }
  \AttributeTok{x =} \FunctionTok{runif}\NormalTok{(}\DecValTok{100}\NormalTok{), }
  \AttributeTok{y =} \FunctionTok{runif}\NormalTok{(}\DecValTok{100}\NormalTok{)}
\NormalTok{)}
\end{Highlighting}
\end{Shaded}
\end{frame}

\begin{frame}[fragile]
We can use \texttt{nest} and \texttt{unnest} to create so called
\texttt{list}-columns.

\begin{Shaded}
\begin{Highlighting}[]
\NormalTok{dat }\SpecialCharTok{|\textgreater{}} \FunctionTok{nest}\NormalTok{(}\AttributeTok{data =} \FunctionTok{c}\NormalTok{(x, y))}
\end{Highlighting}
\end{Shaded}

\begin{verbatim}
# A tibble: 10 x 2
      id data             
   <int> <list>           
 1     1 <tibble [10 x 2]>
 2     2 <tibble [10 x 2]>
 3     3 <tibble [10 x 2]>
 4     4 <tibble [10 x 2]>
 5     5 <tibble [10 x 2]>
 6     6 <tibble [10 x 2]>
 7     7 <tibble [10 x 2]>
 8     8 <tibble [10 x 2]>
 9     9 <tibble [10 x 2]>
10    10 <tibble [10 x 2]>
\end{verbatim}
\end{frame}

\begin{frame}[fragile]
\begin{Shaded}
\begin{Highlighting}[]
\NormalTok{dat }\SpecialCharTok{|\textgreater{}} \FunctionTok{nest}\NormalTok{(}\AttributeTok{data =} \SpecialCharTok{{-}}\NormalTok{id)}
\end{Highlighting}
\end{Shaded}

\begin{verbatim}
# A tibble: 10 x 2
      id data             
   <int> <list>           
 1     1 <tibble [10 x 2]>
 2     2 <tibble [10 x 2]>
 3     3 <tibble [10 x 2]>
 4     4 <tibble [10 x 2]>
 5     5 <tibble [10 x 2]>
 6     6 <tibble [10 x 2]>
 7     7 <tibble [10 x 2]>
 8     8 <tibble [10 x 2]>
 9     9 <tibble [10 x 2]>
10    10 <tibble [10 x 2]>
\end{verbatim}
\end{frame}

\begin{frame}[fragile]
We can then work on the nested column(s), using \texttt{mutate} in
combination with \texttt{map\_*}:

\begin{Shaded}
\begin{Highlighting}[]
\NormalTok{dat }\SpecialCharTok{|\textgreater{}} \FunctionTok{nest}\NormalTok{(}\AttributeTok{data =} \SpecialCharTok{{-}}\NormalTok{id) }\SpecialCharTok{|\textgreater{}} 
  \FunctionTok{mutate}\NormalTok{(}\AttributeTok{centroid.x =} \FunctionTok{map\_dbl}\NormalTok{(data, }\SpecialCharTok{\textasciitilde{}} \FunctionTok{mean}\NormalTok{(.x}\SpecialCharTok{$}\NormalTok{x)))}
\end{Highlighting}
\end{Shaded}

\begin{verbatim}
# A tibble: 10 x 3
      id data              centroid.x
   <int> <list>                 <dbl>
 1     1 <tibble [10 x 2]>      0.315
 2     2 <tibble [10 x 2]>      0.444
 3     3 <tibble [10 x 2]>      0.410
 4     4 <tibble [10 x 2]>      0.672
 5     5 <tibble [10 x 2]>      0.459
 6     6 <tibble [10 x 2]>      0.562
 7     7 <tibble [10 x 2]>      0.519
 8     8 <tibble [10 x 2]>      0.575
 9     9 <tibble [10 x 2]>      0.418
10    10 <tibble [10 x 2]>      0.422
\end{verbatim}
\end{frame}

\begin{frame}[fragile]
\begin{block}{Lets come back to our example}
\protect\hypertarget{lets-come-back-to-our-example}{}
First, lets use \texttt{nest()}:

\begin{Shaded}
\begin{Highlighting}[]
\NormalTok{amt\_fisher }\SpecialCharTok{|\textgreater{}} \FunctionTok{nest}\NormalTok{(}\AttributeTok{data =} \SpecialCharTok{{-}}\NormalTok{name)}
\end{Highlighting}
\end{Shaded}

\begin{verbatim}
# A tibble: 4 x 2
  name    data                  
  <chr>   <list>                
1 Leroy   <trck_xyt [919 x 5]>  
2 Ricky T <trck_xyt [8,958 x 5]>
3 Lupe    <trck_xyt [3,004 x 5]>
4 Lucile  <trck_xyt [1,349 x 5]>
\end{verbatim}
\end{block}
\end{frame}

\begin{frame}[fragile]
Now we can iterate over all animals

\begin{Shaded}
\begin{Highlighting}[]
\NormalTok{amt\_fisher }\SpecialCharTok{|\textgreater{}} \FunctionTok{nest}\NormalTok{(}\AttributeTok{data =} \SpecialCharTok{{-}}\NormalTok{name) }\SpecialCharTok{|\textgreater{}} 
  \FunctionTok{mutate}\NormalTok{(}\AttributeTok{mean.sl =} \FunctionTok{map\_dbl}\NormalTok{(data, }\SpecialCharTok{\textasciitilde{}} \FunctionTok{step\_lengths}\NormalTok{(.x) }\SpecialCharTok{|\textgreater{}} 
                             \FunctionTok{mean}\NormalTok{(}\AttributeTok{na.rm =} \ConstantTok{TRUE}\NormalTok{)))}
\end{Highlighting}
\end{Shaded}

\begin{verbatim}
# A tibble: 4 x 3
  name    data                   mean.sl
  <chr>   <list>                   <dbl>
1 Leroy   <trck_xyt [919 x 5]>     187. 
2 Ricky T <trck_xyt [8,958 x 5]>    42.0
3 Lupe    <trck_xyt [3,004 x 5]>    61.0
4 Lucile  <trck_xyt [1,349 x 5]>    86.0
\end{verbatim}
\end{frame}

\begin{frame}[fragile]
The approach of list-columns makes it easy to have several grouping
instances.

\begin{Shaded}
\begin{Highlighting}[]
\NormalTok{amt\_fisher }\SpecialCharTok{|\textgreater{}} \FunctionTok{mutate}\NormalTok{(}\AttributeTok{wk =}\NormalTok{ lubridate}\SpecialCharTok{::}\FunctionTok{week}\NormalTok{(t\_)) }\SpecialCharTok{|\textgreater{}} 
  \FunctionTok{nest}\NormalTok{(}\AttributeTok{data =} \SpecialCharTok{{-}}\FunctionTok{c}\NormalTok{(name, wk)) }\SpecialCharTok{|\textgreater{}} 
  \FunctionTok{mutate}\NormalTok{(}\AttributeTok{mean.sl =} \FunctionTok{map\_dbl}\NormalTok{(data, }\SpecialCharTok{\textasciitilde{}} \FunctionTok{step\_lengths}\NormalTok{(.x) }\SpecialCharTok{|\textgreater{}} 
                             \FunctionTok{mean}\NormalTok{(}\AttributeTok{na.rm =} \ConstantTok{TRUE}\NormalTok{)))}
\end{Highlighting}
\end{Shaded}

\begin{verbatim}
# A tibble: 21 x 4
   name       wk data                   mean.sl
   <chr>   <dbl> <list>                   <dbl>
 1 Leroy       6 <trck_xyt [47 x 5]>       12.6
 2 Leroy       7 <trck_xyt [330 x 5]>     204. 
 3 Leroy       8 <trck_xyt [261 x 5]>     239. 
 4 Leroy       9 <trck_xyt [281 x 5]>     145. 
 5 Ricky T     6 <trck_xyt [232 x 5]>      40.6
 6 Ricky T     7 <trck_xyt [737 x 5]>      40.3
 7 Ricky T     8 <trck_xyt [1,345 x 5]>    46.1
 8 Ricky T     9 <trck_xyt [1,488 x 5]>    47.1
 9 Ricky T    10 <trck_xyt [1,290 x 5]>    42.3
10 Ricky T    11 <trck_xyt [1,280 x 5]>    36.2
# i 11 more rows
\end{verbatim}
\end{frame}

\begin{frame}{Take-home messages}
\protect\hypertarget{take-home-messages}{}
\begin{enumerate}
\tightlist
\item
  Available data structures can be used to deal with multiple animals.
\item
  Lists are just ``containers'' that contain objects.
\item
  List columns are a great way to organize complex data structures.
\end{enumerate}
\end{frame}


\end{document}
